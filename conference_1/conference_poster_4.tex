%%%%%%%%%%%%%%%%%%%%%%%%%%%%%%%%%%%%%%%%%
% baposter Landscape Poster
% LaTeX Template
% Version 1.0 (11/06/13)
%
% baposter Class Created by:
% Brian Amberg (baposter@brian-amberg.de)
%
% This template has been downloaded from:
% http://www.LaTeXTemplates.com
%
% License:
% CC BY-NC-SA 3.0 (http://creativecommons.org/licenses/by-nc-sa/3.0/)
%
%%%%%%%%%%%%%%%%%%%%%%%%%%%%%%%%%%%%%%%%%


%----------------------------------------------------------------------------------------
%	PACKAGES AND OTHER DOCUMENT CONFIGURATIONS
%----------------------------------------------------------------------------------------

\documentclass[landscape,a0paper,fontscale=0.285]{baposter} % Adjust the font scale/size here

\usepackage{graphicx} % Required for including images
\graphicspath{{figures/}} % Directory in which figures are stored

\usepackage{amsmath} % For typesetting math
\usepackage{amssymb} % Adds new symbols to be used in math mode

\usepackage{booktabs} % Top and bottom rules for tables
\usepackage{enumitem} % Used to reduce itemize/enumerate spacing
\usepackage{palatino} % Use the Palatino font
\usepackage[font=small,labelfont=bf]{caption} % Required for specifying captions to tables and figures

\usepackage{multicol} % Required for multiple columns
\setlength{\columnsep}{1.5em} % Slightly increase the space between columns
\setlength{\columnseprule}{0mm} % No horizontal rule between columns

\usepackage{tikz} % Required for flow chart
\usetikzlibrary{shapes,arrows} % Tikz libraries required for the flow chart in the template

\newcommand{\compresslist}{ % Define a command to reduce spacing within itemize/enumerate environments, this is used right after \begin{itemize} or \begin{enumerate}
\setlength{\itemsep}{1pt}
\setlength{\parskip}{0pt}
\setlength{\parsep}{0pt}
}

\definecolor{lightblue}{rgb}{0.145,0.6666,1} % Defines the color used for content box headers

\begin{document}

\begin{poster}
{
headerborder=closed, % Adds a border around the header of content boxes
colspacing=1em, % Column spacing
bgColorOne=white, % Background color for the gradient on the left side of the poster
bgColorTwo=white, % Background color for the gradient on the right side of the poster
borderColor=lightblue, % Border color
headerColorOne=black, % Background color for the header in the content boxes (left side)
headerColorTwo=lightblue, % Background color for the header in the content boxes (right side)
headerFontColor=white, % Text color for the header text in the content boxes
boxColorOne=white, % Background color of the content boxes
textborder=roundedleft, % Format of the border around content boxes, can be: none, bars, coils, triangles, rectangle, rounded, roundedsmall, roundedright or faded
eyecatcher=true, % Set to false for ignoring the left logo in the title and move the title left
headerheight=0.1\textheight, % Height of the header
headershape=roundedright, % Specify the rounded corner in the content box headers, can be: rectangle, small-rounded, roundedright, roundedleft or rounded
headerfont=\Large\bf\textsc, % Large, bold and sans serif font in the headers of content boxes
%textfont={\setlength{\parindent}{1.5em}}, % Uncomment for paragraph indentation
linewidth=2pt % Width of the border lines around content boxes
}
%----------------------------------------------------------------------------------------
%	TITLE SECTION 
%----------------------------------------------------------------------------------------
%
{\includegraphics[height=4em]{mohit}} % First university/lab logo on the left
{\bf\textsc{Autonomous and Augmented Vehicle Security}\vspace{0.5em}} % Poster title
{\textsc{\normalsize{ Kevin Gilbert, Christopher Haster, Gilberto Rodriguez III, Hao Chen, Young Chou, Joshua Bryant\\ } \hspace{12pt} University of Texas at Austin Cockrell School of Engineering}} % Author names and institution
{\includegraphics[height=4em]{ti_logo}} % Second university/lab logo on the right

%----------------------------------------------------------------------------------------
%	OBJECTIVES
%----------------------------------------------------------------------------------------

\headerbox{Objectives}{name=objectives,column=0,row=0}{

Our research project is focused on highlighting security concerns in augmented and autonomous vehicles. We have developed and built a robotics testbed and simulator on which we can measure and apply real-world data. We primarily focus on the two coupled weak points in augmented automotive cybersecurity: wireless transceiver entry points into an unsecured Controller Area Network (CAN).


\vspace{0.3em} % When there are two boxes, some whitespace may need to be added if the one on the right has more content
}

%----------------------------------------------------------------------------------------
%	INTRODUCTION
%----------------------------------------------------------------------------------------

\headerbox{Introduction}{name=introduction,column=1,row=0,bottomaligned=objectives}{

Aliquam non lacus dolor, \textit{a aliquam quam}. Cum sociis natoque penatibus et magnis dis parturient montes, nascetur ridiculus mus. Nulla in nibh mauris. Donec vel ligula nisi, a lacinia arcu. Sed mi dui, malesuada vel consectetur et, egestas porta nisi. Sed eleifend pharetra dolor, et dapibus est vulputate eu. \textbf{Integer faucibus elementum felis vitae fringilla.} In hac habitasse platea dictumst.
}

%----------------------------------------------------------------------------------------
%	Measurements
%----------------------------------------------------------------------------------------
\headerbox{Measurements}{name=measurements,column=0,span=2} {
Fuck this shit
}


%----------------------------------------------------------------------------------------
%	Modules
%----------------------------------------------------------------------------------------

\headerbox{Modules}{name=modules,column=1,span=1,row=0}{

%\begin{multicols}{2}
\vspace{1em}
%\begin{center}
%\includegraphics[width=0.8\linewidth]{UART-CAN_PacketTranslation}
%\captionof{figure}{FPGA Bus Capture}
%\end{center}

Our primary modules were broken down into: 

\begin{enumerate}\compresslist
\item FPGA -
\subitem CAN Bus
\subitem UART - CAN Packet Translation
\subitem PWM Generation
\subitem Hardware Encryption
\item Wireless Transceivers -
\subitem Data Transmission
\subitem Software Encryption
\item Robotics Testbed -
\subitem Data Measurement
\item Embedded -
\subitem IMU Measurement
\subitem Motor Control (PID)
\subitem Laptop to CAN Bus Interface
\subitem Sensor Interface
\item Simulator -
\subitem Network Timing Constraints
\subitem Large Data Generation
\end{enumerate}

%\end{multicols}

%------------------------------------------------

%\begin{multicols}{2}
%\vspace{1em}
%Each module was designed and tested on an individual basis before being combined into our end %product. The attached images show captured data from our CAN bus, IMU, and simulator.

%\begin{center}
%\includegraphics[width=0.8\linewidth]{rand_plot}
%\captionof{figure}{IMU data (random)}
%\end{center}

%\end{multicols}
}

%----------------------------------------------------------------------------------------
%	REFERENCES
%----------------------------------------------------------------------------------------

\headerbox{References}{name=references,column=0,above=bottom}{

\renewcommand{\section}[2]{\vskip 0.05em} % Get rid of the default "References" section title
\nocite{*} % Insert publications even if they are not cited in the poster
\small{ % Reduce the font size in this block
\bibliographystyle{unsrt}
\bibliography{sample} % Use sample.bib as the bibliography file
}}

%----------------------------------------------------------------------------------------
%	FUTURE RESEARCH
%----------------------------------------------------------------------------------------

\headerbox{Future Research}{name=futureresearch,column=0,span=2,aligned=references,above=bottom}{ % This block is as tall as the references block

\begin{multicols}{2}
Integer sed lectus vel mauris euismod suscipit. Praesent a est a est ultricies pellentesque. Donec tincidunt, nunc in feugiat varius, lectus lectus auctor lorem, egestas molestie risus erat ut nibh.

Maecenas viverra ligula a risus blandit vel tincidunt est adipiscing. Suspendisse mollis iaculis sem, in \emph{imperdiet} orci porta vitae. Quisque id dui sed ante sollicitudin sagittis.
\end{multicols}
}

%----------------------------------------------------------------------------------------
%	CONTACT INFORMATION
%----------------------------------------------------------------------------------------

%\headerbox{Contact Information}{name=contact,column=3,aligned=references,above=bottom}{ % This %block is as tall as the references block

%\begin{description}\compresslist
%\item[Web] www.university.edu/smithlab
%\item[Email] john@smith.com
%\item[Phone] +1 (000) 111 1111
%\end{description}
%}

%----------------------------------------------------------------------------------------
%	CONCLUSION
%----------------------------------------------------------------------------------------

\headerbox{Conclusion}{name=conclusion,column=2,span=2,row=0,below=modules}{

\begin{multicols}{2}

\tikzstyle{decision} = [diamond, draw, fill=blue!20, text width=4.5em, text badly centered, node distance=2cm, inner sep=0pt]
\tikzstyle{block} = [rectangle, draw, fill=blue!20, text width=5em, text centered, rounded corners, minimum height=4em]
\tikzstyle{line} = [draw, -latex']
\tikzstyle{cloud} = [draw, ellipse, fill=red!20, node distance=3cm, minimum height=2em]

\begin{tikzpicture}[node distance = 2cm, auto]
\node [block] (init) {Initialize Model};
\node [cloud, left of=init] (Start) {Start};
\node [cloud, right of=init] (Start2) {Start Two};
\node [block, below of=init] (init2) {Initialize Two};
\node [decision, below of=init2] (End) {End};
\path [line] (init) -- (init2);
\path [line] (init2) -- (End);
\path [line, dashed] (Start) -- (init);
\path [line, dashed] (Start2) -- (init);
path [line, dashed] (Start2) |- (init2);
\end{tikzpicture}

%------------------------------------------------

\begin{itemize}\compresslist
\item Pellentesque eget orci eros. Fusce ultricies, tellus et pellentesque fringilla, ante massa luctus libero, quis tristique purus urna nec nibh. Phasellus fermentum rutrum elementum. Nam quis justo lectus.
\item Vestibulum sem ante, hendrerit a gravida ac, blandit quis magna.
\item Donec sem metus, facilisis at condimentum eget, vehicula ut massa. Morbi consequat, diam sed convallis tincidunt, arcu nunc.
\item Nunc at convallis urna. isus ante. Pellentesque condimentum dui. Etiam sagittis purus non tellus tempor volutpat. Donec et dui non massa tristique adipiscing.
\end{itemize}

\end{multicols}
}

%----------------------------------------------------------------------------------------
%	MATERIALS AND METHODS
%----------------------------------------------------------------------------------------

\headerbox{Materials \& Methods}{name=method,column=0,below=objectives,bottomaligned=modules}{ % This block's bottom aligns with the bottom of the conclusion block

The following materials were required to complete the research:



The following equations were used for statistical analysis:

}

%----------------------------------------------------------------------------------------
%	DATA (RESULTS 2)
%----------------------------------------------------------------------------------------

\headerbox{Data}{name=results2,column=2, span=2,above=futureresearch}{ % This block's bottom aligns with the bottom of the conclusion block

\begin{multicols}{2}
Donec faucibus purus at tortor egestas eu fermentum dolor facilisis. Maecenas tempor dui eu neque fringilla rutrum. Mauris \emph{lobortis} nisl accumsan.

\includegraphics[width=0.9\linewidth]{rand_plot}

Donec faucibus purus at tortor egestas eu fermentum dolor facilisis. Maecenas tempor dui eu neque fringilla rutrum. Mauris \emph{lobortis} nisl accumsan.

\includegraphics[width=0.9\linewidth]{UART-CAN_PacketTranslation}

Nulla ut porttitor enim. Suspendisse venenatis dui eget eros gravida tempor. Mauris feugiat elit et augue placerat ultrices. Morbi accumsan enim nec tortor consectetur non commodo.

\includegraphics[width=0.9\linewidth]{car_sim}

Nulla ut porttitor enim. Suspendisse venenatis dui eget eros gravida tempor. Mauris feugiat elit et augue placerat ultrices. Morbi accumsan enim nec tortor consectetur non commodo.

\begin{center}
\begin{tabular}{l l l}
\toprule
\textbf{Treatments} & \textbf{Response 1} & \textbf{Response 2}\\
\midrule
Treatment 1 & 0.0003262 & 0.562 \\
Treatment 2 & 0.0015681 & 0.910 \\
Treatment 3 & 0.0009271 & 0.296 \\
\bottomrule
\end{tabular}
\captionof{table}{Table caption}
\end{center}
\end{multicols}
}

%----------------------------------------------------------------------------------------

\end{poster}

\end{document}